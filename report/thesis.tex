\PassOptionsToPackage{authoryear,round}{natbib}
\documentclass[11pt]{article}
\usepackage{thesis}

\usepackage{hyperref}       % hyperlinks
\usepackage{url}            % simple URL typesetting
\usepackage{booktabs}       % professional-quality tables
\usepackage{amsfonts}       % blackboard math symbols
\usepackage{nicefrac}       % compact symbols for 1/2, etc.
\usepackage{microtype}      % microtypography
\usepackage{graphicx}
\usepackage{amsmath}
\usepackage{titling}
\usepackage{caption}
\captionsetup{labelsep=period}
%\usepackage[font=small]{caption}
\setlength{\droptitle}{-40pt}  % Adjust as needed

\usepackage{titlesec}

\titlespacing{\section}
  {0pt}    % left margin
  {0.7ex plus 0.5ex minus .2ex}  % space before section title
  {0.7ex plus .2ex}  % space after section title (before paragraph)

\titlespacing{\subsection}
  {0pt}
  {1.2ex plus 0.3ex minus .2ex}
  {0.6ex plus .1ex}

\title{\vspace{-1em}{\Large\textbf{Monocular 4D Reconstruction of (in-the-wild) Scenes with Multiple People}}\vspace{-1em}}

\author{
  \begin{tabular}{c c}
  Ludek Cizinsky \\
  \texttt{ludek.cizinsky@epfl.ch} \\
  EPFL
  \end{tabular}
}

\begin{document}

\date{}
\maketitle

\begin{abstract}
% Brief summary of the problem, method, and results (to be written last)
\end{abstract}


\section{Introduction}
% Problem context and motivation
% - I see two important applications of my work
% - First: interactive media - we no longer have to rely on watching monocular feed, and instead can view the given video from any angle we want. This is especially useful in sports broadcasting, where the viewer can choose their own perspective. Here, the important aspect of the reconstruction is how accurate the extracted motion is, and how realistic the novel view renderings are.
% - Second: With the recent advances in humanoid robotics, being able to precisely recover human motion from monocular videos can help extract training motion data for robots to imitate. Here, we only care about the accuracy of the recovered motion, and not so much about the visual quality of the renderings.
% Challenges in monocular dynamic scene reconstruction
% - General challenges in dynamic scene reconstruction
% -- In general, 4D reconstruction from monocular video is highly ill-posed problem
% -- Apart from obtaining multi-view consistency from monocular video, we also have to ensure temporal consistency
% -- In addition, the system needs to be able to accurately disentangle the motion of camera and objects in the scene
% - Specific challenges in human-centric reconstruction
% -- Humans are highly non-rigid objects, with complex articulations and deformations including clothing dynamics and hair motion
% -- Human motion is often fast and unpredictable, leading to motion blur and occlusions
% -- when we have multiple people in the scene, we have to deal with inter-person occlusions and interactions  
% Gap in existing methods
% - Majority of the existing human-centric scenes methods focues on mapping either single image, set of image or monocular video to parametric human model (SMPL, SMPL-X, etc.). These approaches assume clean video capture conditions and fail for the in the wild scenarios where we might have multiple people interacting and occluding each other. 
% - In the last year or two, there has been a new wave of papers that deviate from the tradional 3D reconstruction methods, and instead, train a feedforward network to directly map the input image or video to the target set of modalities, usually depth maps and camera parameters. The main limitation of these approaches is that while at a first glance they give decent predictions, they are still much less acurrate than their more classical optimisation based counterparts. 
% - While there have been previous attempts for monocular 4D reconstruction of dynamic human centric scenes, the main limitation of these approaches is that they require order of hours to days of training time per scene, making them impractical for real world applications. 
% Contributions
% - The main contribution of this work is a novel hybrid framework for monocular 4D reconstruction of dynamic human centric scenes that combines the best of both worlds - the speed and efficiency of feedforward networks, and the accuracy and quality of optimisation based approaches.
% - As a result, we obtain the quality comparable to the state of the art methods that require hours to days of training time, while being able to reconstruct a scene in order of minutes.

\section{Related Work}
\subsection{3D Gaussian Splatting}
\subsection{Dynamic Scene Reconstruction}
\subsection{Human-Centric Reconstruction}
\subsection{Novel View Synthesis}
\subsection{Sports and Monocular Reconstruction}

\section{Problem Definition}
% - Input and output specification
% - Assumptions
% - Scope and evaluation goals

\section{Method}
\subsection{Overview}
\subsection{Preprocessing}
\subsection{Camera Setup and Coordinate Systems}
\subsection{Human Representation}
\subsection{Static and Dynamic Gaussian Modeling}
\subsection{Optimization and Training Procedure}
\subsection{Novel View Rendering}
% \subsection{Optional Refinement Stage}

\section{Experimental Setup}
\subsection{Datasets}
\subsection{Baselines}
\subsection{Evaluation Metrics}
\subsection{Implementation Details}

\section{Results}
\subsection{Quantitative Results}
\subsection{Qualitative Results}
\subsection{Ablation Studies}

\section{Discussion}
% - Interpretation of results
% - Strengths and weaknesses
% - Comparison to baselines

\section{Limitations}
% - Failure cases
% - Assumptions that do not hold
% - Scalability and generalization limits

\section{Conclusion}
% - Summary of contributions
% - Key findings
% - Future work

\bibliographystyle{ieee_fullname}
\bibliography{references}

\bibliographystyle{plainnat}
\bibliography{references}

\end{document}